\documentclass[a4j, dvipdfmx, twocolumn]{jsarticle}

\usepackage[dvipdfmx]{graphicx}
\usepackage[dvipdfmx]{color}
\usepackage{amsmath, amssymb}  % Extend math
\usepackage{bm}
\usepackage{float}  % Improved interface for floating objects
\usepackage{multirow}  % Create complex table
\usepackage{url}  % Display URL
\usepackage{subcaption}  % Use subcaption
\usepackage{tabularx}  % Newline in table cell

\usepackage[final]{listings}

\lstset{
  language=python,
  basicstyle=\ttfamily\scriptsize,
  commentstyle={\ttfamily \color[cmyk]{0,1,1,0}},
  keywordstyle={\bfseries \color[cmyk]{1,1,0,0}},
  stringstyle={\ttfamily \color[cmyk]{1,0,1,0.5}},
  stepnumber=1,
  numberstyle=\ttfamily,
  breaklines=true,
  breakindent=20pt,
  frame=tblr,
  framesep=4pt,
  tabsize=2
}

\newcommand{\figref}[1]{Fig.\ref{#1}}
\newcommand{\tabref}[1]{Table.\ref{#1}}
\newcommand{\secref}[1]{\ref{#1}節}
\renewcommand{\figurename}{Fig.}
\renewcommand{\tablename}{Table.}


\title{主専攻実験A 最終レポート}
\author{岡部 純弥}
\date{\today}

\begin{document}

\twocolumn[
\maketitle
\begin{abstract}
  本課題では,Googleの検索アルゴリズムとして非常に有名なPageRankアルゴリズム\cite{ilprints361}\cite{ilprints422}の理論を理解し,これを用いた計算機実験を行った.実際に,日本国内の主要空港間の移動者数データに対してPageRankアルゴリズムを適用し,各空港の重要度を計算した.
  % さらに計算結果から,航空ネットワークが中央集権的である\footnote{一般的に,特定のいくつかの空港に路線が集中するようなネットワークを\emph{ハブアンドスポークシステム}という.一方で非中央集権的な,各空港間に直行便が運行しているシステムを\emph{ポイントトゥポイントシステム}という.}べきか否かを考察した.
\end{abstract}
]

\section{はじめに}
\subsection{PageRankとは}
PageRankとは,Brin, Page\cite{ilprints361}によって提案されたGoogleの検索システムで用いられているアルゴリズムである.PageRankでは,Webページ間のハイパーリンク関係を用いて,各ページの重要度を計算する.これは,\emph{良いWebページは別の良いWebページからリンクされている}という考え方をもとに実現されている.Facts about Google and Competition
\footnote{\url{https://web.archive.org/web/20111104131332/https://www.google.com/competition/howgooglesearchworks.html}}
によると,
\begin{quote}
  PageRank works by counting the number and quality of links to a page to determine a rough estimate of how important the website is. The underlying assumption is that more important websites are likely to receive more links from other websites
\end{quote}
と確かに記載されている.またこの考え方は,論文の引用/被引用数ネットワークや共著ネットワークと非常に似ている.つまり良質な論文は,別な良質な論文からリンクされているという考え方である.実際にPageRankアルゴリズムを用いた論文の共著システムに関する研究として,Ma et al.\cite{ma2008bringing},Ding et al.\cite{ding2009pagerank} などが挙げられる.

\subsection{応用先}
PageRankは,Webサイトの重要度付けの他にも,(ソーシャル)ネットワーク分析,物理学,化学,生物学など多数の応用先がある.ソーシャルネットワーク分析の事例としてはBahmani et al.\cite{bahmani2010fast}などが挙げられる.またPageRankの応用に関する総説論文としてはGleich\cite{gleich2015pagerank},Berkhin\cite{berkhin2005survey}が著名である.

\section{PageRank}
\subsection{定義}
ここでは基本的な
\footnote{Page et al.\cite{ilprints422}の論文に基づいた}
PageRankのアルゴリズムを紹介する.

$u$ をあるWebページとする.また,$u$ \emph{から}リンクするWebページの集合を $F_u$,$u$ \emph{に}リンクするWebページの集合を $B_u$ とする.さらに,$F_u$ の要素数 $N_u$
\footnote{すなわち $N_u = |F_u|$},正規化するための定数 $c$ を用いると,$u$ のランク $R(u)$ は式\eqref{eq:rank1}によって定義される.
\begin{equation}
  \label{eq:rank1}
  R(u) = c  \sum_{v \in B_u} \frac{R(v)}{N_v}
\end{equation}
$R(v) / N_v$ は,$v$ のランクを $F_u$ の要素数,すなわち $v$ からリンクするページの総数で割ったものである.つまり,$R(u)$ は $u$ にリンクするすべてのページに対して $R(v) / N_v$ を計算し,その総和に $c$ を掛けたものである.したがって,ランクの高いページからリンクされているページもまたランクが高くなる傾向にある.

式\eqref{eq:rank1}を別の観点から評価し直してみる.
ある正方行列 $A$ を考え,$A$ の $(u,\,v)$ 成分を
\begin{equation}
  \label{eq:Auv}
  A_{u,v} = \begin{cases}
    1/N_{u,v} & \text{if edge from u to v exists} \\
    0 & \text{otherewise} \\
  \end{cases}
\end{equation}
と定義する.このとき,$R$ をベクトルとして考えると
\begin{equation}
  \label{eq:R}
  R = cAR
\end{equation}
と表すことができる.これは $R$ が $A$ の固有ベクトルに他ならないことを示している.\footnote{さらにこのときの固有値は $c$ である.}

しかし式\eqref{eq:rank1}の定義には少し問題がある.ある2つのページ $u'$ と $v'$ が相互にリンクしており,なおかつ他のどのページともリンクしない状況を考えてみる.さらに,別のあるページが $u'$ あるいは $v'$ にリンクしているものとする.このとき,ランクをうまく配分することができない.そこで,式\eqref{eq:rank1}の定義を,あるベクトル $E(u)$ \footnote{$E(u)$ はWeb上のランクのソースに対応している} を用いて式\eqref{eq:rank2}に再度定義し直す.
\begin{equation}
  \label{eq:rank2}
  R'(u) = c \left( \sum_{v \in B_u} \frac{R'(v)}{N_v} + E(u)\right)
\end{equation}
ただし,式\eqref{eq:rank2} において $||R'||_1 = 1$ 
\footnote{$||R'||_1$ は,$R'$ の $L_1$ 正規化ノルムを表す.}
を満たすものとする.式\eqref{eq:rank2}は $cE(u)$ の項によって正則化されているため,前述したような問題が起きることはない.以後,この定義を用いて議論を進める.

PageRankのより詳細な理論,およびその拡張に関しては,Page et al.\cite{ilprints422},Bianchini et al.\cite{bianchini2005inside},Langville, Meyer\cite{langville2004deeper}などを参照されたい.

\section{計算機実験}
本課題では e-stat
\footnote{政府統計の総合窓口}
\footnote{\url{https://www.e-stat.go.jp}}
上で入手できる,日本国内の主要空港
\footnote{東京国際(羽田),成田国際,新千歳,大阪国際(伊丹),関西国際,福岡,那覇の7空港}
間の令和2年2月の月間移動者数の旅客数を用いた.このデータでは,各ODペアに対する月間の旅客移動数が記載されている.\footnote{ただし,羽田-成田間,伊丹-関西間のデータは見つからなかったため,0人として扱っている.}

\section{結果}


\begin{table}[htb]
    \begin{tabular}{cl}
      \hline
      空港 & 重要度 \\
      \hline
      \hline
      羽田 & 0.295 \\
      成田 & 0.083 \\
      新千歳 & 0.147 \\
      伊丹 & 0.101 \\
      関西 & 0.076 \\
      福岡 & 0.160 \\
      那覇 & 0.136 \\
      \hline
    \end{tabular}
  \centering
  \caption{各空港の重要度}
\end{table}

\section{考察}

\begin{table}[htb]
  \begin{tabular}{cl}
    \hline
    空港 & 旅客数 \\
    \hline
    \hline
    羽田 & 20,606,398 \\
    成田 & 1,984,001 \\
    新千歳 & 6,436,335 \\
    伊丹 & 5,812,333 \\
    関西 & 2,051,220 \\
    福岡 & 6,485,437 \\
    那覇 & 6,588,217 \\
    \hline
  \end{tabular}
\centering
\caption{令和2年度 年間旅客数(国内)}
\end{table}


\begin{table}[htb]
  \begin{tabular}{cll}
    \hline
    空港 & 重要度順位 & 旅客数順位 \\
    \hline
    \hline
    羽田 & 1 & 1 \\
    成田 & 6 & 7 \\
    新千歳 & 3 & 4 \\
    伊丹 & 5 & 5 \\
    関西 & 7 & 6 \\
    福岡 & 2 & 3 \\
    那覇 & 4 & 2 \\
    \hline
  \end{tabular}
\centering
\caption{}
\end{table}

\section{まとめ}


\bibliographystyle{unsrt}
\bibliography{bib/ref}

\end{document}