\documentclass[a4j, dvipdfmx, twocolumn]{jsarticle}

\usepackage[dvipdfmx]{graphicx}
\usepackage[dvipdfmx]{color}
\usepackage{amsmath, amssymb}  % Extend math
\usepackage{bm}
\usepackage{float}  % Improved interface for floating objects
\usepackage{multirow}  % Create complex table
\usepackage{url}  % Display URL
\usepackage{subcaption}  % Use subcaption
\usepackage{tabularx}  % Newline in table cell

\usepackage[final]{listings}

\lstset{
  language=python,
  basicstyle=\ttfamily\scriptsize,
  commentstyle={\ttfamily \color[cmyk]{0,1,1,0}},
  keywordstyle={\bfseries \color[cmyk]{1,1,0,0}},
  stringstyle={\ttfamily \color[cmyk]{1,0,1,0.5}},
  stepnumber=1,
  numberstyle=\ttfamily,
  breaklines=true,
  breakindent=20pt,
  frame=tblr,
  framesep=4pt,
  tabsize=2
}

\newcommand{\figref}[1]{Fig.\ref{#1}}
\newcommand{\tabref}[1]{Table.\ref{#1}}
\newcommand{\secref}[1]{\ref{#1}節}
\renewcommand{\figurename}{Fig.}
\renewcommand{\tablename}{Table.}


\title{主専攻実験A 最終レポート}
\author{岡部 純弥}
\date{\today}
\begin{document}
\maketitle
\begin{abstract}
  本課題では,Googleの検索アルゴリズムとして非常に有名なPageRankアルゴリズム\cite{ilprints422}の理論を理解し,これを用いた計算機実験を行った.実際に,航空ネットワークに関するダミーデータを作成し,PageRankアルゴリズムを用いて,各空港の重要度を計算した.さらに計算結果から,航空ネットワークが中央集権的である\footnote{一般的に,特定のいくつかの空港に路線が集中するようなネットワークを\emph{ハブアンドスポークシステム}という.一方で非中央集権的な,各空港間に直行便が運行しているシステムを\emph{ポイントトゥポイントシステム}という.}べきか否かを考察した.
\end{abstract}

\section{はじめに}
\subsection{PageRankとは}
PageRankとは,Googleの検索システムで用いられているアルゴリズムである.PageRankでは,Webページ間のハイパーリンク関係を用いて,各ページの重要度を計算する.これは,\emph{良いWebページは別の良いWebページからリンクされている}という考え方をもとに実現されている.Facts about Google and Competitionによると,
\begin{quote}
  PageRank works by counting the number and quality of links to a page to determine a rough estimate of how important the website is. The underlying assumption is that more important websites are likely to receive more links from other websites
\end{quote}
と確かに記載されている.またこの考え方は,論文の引用/被引用数ネットワークや共著ネットワークと非常に似ている.つまり良質な論文は,別な良質な論文からリンクされているという考え方である.実際にこの分野に関する研究として,Ma et al.\cite{ma2008bringing},Ding et al.\cite{ding2009pagerank} などが挙げられる.

\subsection{応用先}
PageRankは,Webサイトの重要度付けの他にも,(ソーシャル)ネットワーク分析,物理学,化学,生物学など多数の応用先がある.ソーシャルネットワーク分析の事例としてはBahmani et al.\cite{bahmani2010fast}などが挙げられる.また,PageRankの応用に関する総説論文としてはGleich\cite{gleich2015pagerank},Berkhin\cite{berkhin2005survey}が著名である.


\subsection{アルゴリズム}
ここでは,基本的なPageRankのアルゴリズムを紹介する.


PageRankのより詳細な理論に関しては,Page et al.\cite{ilprints422},Bianchini et al.\cite{bianchini2005inside},Langville, Meyer\cite{langville2004deeper}などを参照されたい.

\subsection{変数の定義}

\subsection{最適化問題の定式化}

\section{計算機実験}

\section{考察}

\section{まとめ}


\bibliographystyle{unsrt}
\bibliography{bib/ref}

\end{document}